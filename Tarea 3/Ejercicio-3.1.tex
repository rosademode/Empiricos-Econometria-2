% Options for packages loaded elsewhere
\PassOptionsToPackage{unicode}{hyperref}
\PassOptionsToPackage{hyphens}{url}
%
\documentclass[
  12pt,
]{article}
\usepackage{amsmath,amssymb}
\usepackage{iftex}
\ifPDFTeX
  \usepackage[T1]{fontenc}
  \usepackage[utf8]{inputenc}
  \usepackage{textcomp} % provide euro and other symbols
\else % if luatex or xetex
  \usepackage{unicode-math} % this also loads fontspec
  \defaultfontfeatures{Scale=MatchLowercase}
  \defaultfontfeatures[\rmfamily]{Ligatures=TeX,Scale=1}
\fi
\usepackage{lmodern}
\ifPDFTeX\else
  % xetex/luatex font selection
\fi
% Use upquote if available, for straight quotes in verbatim environments
\IfFileExists{upquote.sty}{\usepackage{upquote}}{}
\IfFileExists{microtype.sty}{% use microtype if available
  \usepackage[]{microtype}
  \UseMicrotypeSet[protrusion]{basicmath} % disable protrusion for tt fonts
}{}
\makeatletter
\@ifundefined{KOMAClassName}{% if non-KOMA class
  \IfFileExists{parskip.sty}{%
    \usepackage{parskip}
  }{% else
    \setlength{\parindent}{0pt}
    \setlength{\parskip}{6pt plus 2pt minus 1pt}}
}{% if KOMA class
  \KOMAoptions{parskip=half}}
\makeatother
\usepackage{xcolor}
\usepackage[margin=1in]{geometry}
\usepackage{color}
\usepackage{fancyvrb}
\newcommand{\VerbBar}{|}
\newcommand{\VERB}{\Verb[commandchars=\\\{\}]}
\DefineVerbatimEnvironment{Highlighting}{Verbatim}{commandchars=\\\{\}}
% Add ',fontsize=\small' for more characters per line
\usepackage{framed}
\definecolor{shadecolor}{RGB}{248,248,248}
\newenvironment{Shaded}{\begin{snugshade}}{\end{snugshade}}
\newcommand{\AlertTok}[1]{\textcolor[rgb]{0.94,0.16,0.16}{#1}}
\newcommand{\AnnotationTok}[1]{\textcolor[rgb]{0.56,0.35,0.01}{\textbf{\textit{#1}}}}
\newcommand{\AttributeTok}[1]{\textcolor[rgb]{0.13,0.29,0.53}{#1}}
\newcommand{\BaseNTok}[1]{\textcolor[rgb]{0.00,0.00,0.81}{#1}}
\newcommand{\BuiltInTok}[1]{#1}
\newcommand{\CharTok}[1]{\textcolor[rgb]{0.31,0.60,0.02}{#1}}
\newcommand{\CommentTok}[1]{\textcolor[rgb]{0.56,0.35,0.01}{\textit{#1}}}
\newcommand{\CommentVarTok}[1]{\textcolor[rgb]{0.56,0.35,0.01}{\textbf{\textit{#1}}}}
\newcommand{\ConstantTok}[1]{\textcolor[rgb]{0.56,0.35,0.01}{#1}}
\newcommand{\ControlFlowTok}[1]{\textcolor[rgb]{0.13,0.29,0.53}{\textbf{#1}}}
\newcommand{\DataTypeTok}[1]{\textcolor[rgb]{0.13,0.29,0.53}{#1}}
\newcommand{\DecValTok}[1]{\textcolor[rgb]{0.00,0.00,0.81}{#1}}
\newcommand{\DocumentationTok}[1]{\textcolor[rgb]{0.56,0.35,0.01}{\textbf{\textit{#1}}}}
\newcommand{\ErrorTok}[1]{\textcolor[rgb]{0.64,0.00,0.00}{\textbf{#1}}}
\newcommand{\ExtensionTok}[1]{#1}
\newcommand{\FloatTok}[1]{\textcolor[rgb]{0.00,0.00,0.81}{#1}}
\newcommand{\FunctionTok}[1]{\textcolor[rgb]{0.13,0.29,0.53}{\textbf{#1}}}
\newcommand{\ImportTok}[1]{#1}
\newcommand{\InformationTok}[1]{\textcolor[rgb]{0.56,0.35,0.01}{\textbf{\textit{#1}}}}
\newcommand{\KeywordTok}[1]{\textcolor[rgb]{0.13,0.29,0.53}{\textbf{#1}}}
\newcommand{\NormalTok}[1]{#1}
\newcommand{\OperatorTok}[1]{\textcolor[rgb]{0.81,0.36,0.00}{\textbf{#1}}}
\newcommand{\OtherTok}[1]{\textcolor[rgb]{0.56,0.35,0.01}{#1}}
\newcommand{\PreprocessorTok}[1]{\textcolor[rgb]{0.56,0.35,0.01}{\textit{#1}}}
\newcommand{\RegionMarkerTok}[1]{#1}
\newcommand{\SpecialCharTok}[1]{\textcolor[rgb]{0.81,0.36,0.00}{\textbf{#1}}}
\newcommand{\SpecialStringTok}[1]{\textcolor[rgb]{0.31,0.60,0.02}{#1}}
\newcommand{\StringTok}[1]{\textcolor[rgb]{0.31,0.60,0.02}{#1}}
\newcommand{\VariableTok}[1]{\textcolor[rgb]{0.00,0.00,0.00}{#1}}
\newcommand{\VerbatimStringTok}[1]{\textcolor[rgb]{0.31,0.60,0.02}{#1}}
\newcommand{\WarningTok}[1]{\textcolor[rgb]{0.56,0.35,0.01}{\textbf{\textit{#1}}}}
\usepackage{graphicx}
\makeatletter
\def\maxwidth{\ifdim\Gin@nat@width>\linewidth\linewidth\else\Gin@nat@width\fi}
\def\maxheight{\ifdim\Gin@nat@height>\textheight\textheight\else\Gin@nat@height\fi}
\makeatother
% Scale images if necessary, so that they will not overflow the page
% margins by default, and it is still possible to overwrite the defaults
% using explicit options in \includegraphics[width, height, ...]{}
\setkeys{Gin}{width=\maxwidth,height=\maxheight,keepaspectratio}
% Set default figure placement to htbp
\makeatletter
\def\fps@figure{htbp}
\makeatother
\setlength{\emergencystretch}{3em} % prevent overfull lines
\providecommand{\tightlist}{%
  \setlength{\itemsep}{0pt}\setlength{\parskip}{0pt}}
\setcounter{secnumdepth}{-\maxdimen} % remove section numbering
\usepackage{graphicx}
\usepackage{fancyhdr}
\usepackage{mdframed}
\pagestyle{fancy}
\fancyhead{}
\fancyhead[L]{\includegraphics[width=3cm]{logo.jpg}}
\fancyhead[R]{\hspace{0.5cm} Universidad Diego Portales \\ Facultad de Administración y Economía}
\fancypagestyle{plain}{ \fancyhead{} \fancyhead[L]{\includegraphics[width=3cm]{logo.jpg}} \fancyhead[R]{\hspace{0.5 cm} Universidad Diego Portales \\ Facultad de Administración y Economía}}
\ifLuaTeX
  \usepackage{selnolig}  % disable illegal ligatures
\fi
\usepackage{bookmark}
\IfFileExists{xurl.sty}{\usepackage{xurl}}{} % add URL line breaks if available
\urlstyle{same}
\hypersetup{
  pdftitle={Ejercicio 3},
  hidelinks,
  pdfcreator={LaTeX via pandoc}}

\title{\textbf{Ejercicio 3}}
\author{}
\date{\vspace{-2.5em}}

\begin{document}
\maketitle

\maketitle
\vspace{-5em}
\vspace{0.5em}

\begin{center}
\footnotesize \textbf{Curso}: Econometría II \\
\footnotesize \textbf{Profesor}: Mauricio Tejada \\
\footnotesize \textbf{Estudiantes}: Dania Bustamante, Rosana Cardona, José Casanova \\
\footnotesize 15 mayo 2024 \\
\end{center}

Cornwell y Trumbull (1994) utilizaron datos de 90 condados en Carolina
del Norte, de los años 1981 a 1987, para estimar un modelo de efectos
inobservables de la delincuencia. Diversos factores, entre los que se
cuentan la ubicación geográfica, las actitudes hacia la delincuencia,
los registros históricos y las normas de denuncia del crimen, podrían
estar contenidos en el efecto no observable \(a_i\). El modelo propuesto
para explicar el efecto disuasivo de la justicia penal sobre la tasa de
criminalidad es el siguiente:

\hfill\break
\(log(crmrte_{it}) = \beta_0 + \beta_1\log(prbarr_{it}) + \beta_2\log(prbconv_{it}) + \beta_3\log(prbpris_{it})+ \beta_4\log(avgsen_{it}) + \beta_5\log(polpc_{it}) + \beta_6\log(wcon_{it}) + \beta_7\log(wtuc_{it}) + \beta_8\log(wtrd_{it}) + \beta_9\log(wfir_{it}) + \beta_{10}\log(wser_{it}) + \beta_{11}\log(wmfg_{it}) + \beta_{12}\log(wfed_{it}) + \beta_{13}\log(wsta_{it}) + \beta_{14}\log(wloc_{it}) + a_i + \epsilon_{it}\)\\

donde i es el condado y t es el año.

Las variables de salario se incluyen en el modelo para controlar por la
situación económica y por la importancia de los sectores económicos a
nivel de condado. El número de policías (por habitante) se incluye para
controlar por el efecto del control policial.

\subsection{\texorpdfstring{\textbf{\emph{Pregunta
1}}}{Pregunta 1}}\label{pregunta-1}

Estime el modelo por el método de mínimos cuadrados agrupados (pooled
MCO). Estamos interesados en las tres variables de probabilidad
(arresto, condena y tiempo cumplido en prisión) que son las variables
asociadas a la justicia penal. ¿Por qué es probable que los estimadores
obtenidos sean sesgados?

\begin{Shaded}
\begin{Highlighting}[]
\NormalTok{pooled }\OtherTok{\textless{}{-}} \FunctionTok{plm}\NormalTok{(}\FunctionTok{log}\NormalTok{(crmrte) }\SpecialCharTok{\textasciitilde{}} \FunctionTok{log}\NormalTok{(prbarr) }\SpecialCharTok{+} \FunctionTok{log}\NormalTok{(prbconv) }\SpecialCharTok{+} \FunctionTok{log}\NormalTok{(prbpris) }\SpecialCharTok{+} 
                \FunctionTok{log}\NormalTok{(avgsen) }\SpecialCharTok{+} \FunctionTok{log}\NormalTok{(polpc) }\SpecialCharTok{+} \FunctionTok{log}\NormalTok{(wcon) }\SpecialCharTok{+} \FunctionTok{log}\NormalTok{(wtuc) }\SpecialCharTok{+} 
                \FunctionTok{log}\NormalTok{(wtrd) }\SpecialCharTok{+} \FunctionTok{log}\NormalTok{(wfir) }\SpecialCharTok{+} \FunctionTok{log}\NormalTok{(wser) }\SpecialCharTok{+} \FunctionTok{log}\NormalTok{(wmfg) }\SpecialCharTok{+} \FunctionTok{log}\NormalTok{(wfed) }\SpecialCharTok{+} 
                \FunctionTok{log}\NormalTok{(wsta) }\SpecialCharTok{+} \FunctionTok{log}\NormalTok{(wloc), }\AttributeTok{data =}\NormalTok{ panel, }\AttributeTok{model =} \StringTok{"pooling"}\NormalTok{)}

\FunctionTok{summary}\NormalTok{(pooled)}
\end{Highlighting}
\end{Shaded}

\begin{verbatim}
Pooling Model

Call:
plm(formula = log(crmrte) ~ log(prbarr) + log(prbconv) + log(prbpris) + 
    log(avgsen) + log(polpc) + log(wcon) + log(wtuc) + log(wtrd) + 
    log(wfir) + log(wser) + log(wmfg) + log(wfed) + log(wsta) + 
    log(wloc), data = panel, model = "pooling")

Balanced Panel: n = 90, T = 7, N = 630

Residuals:
     Min.   1st Qu.    Median   3rd Qu.      Max. 
-1.825262 -0.192315  0.015372  0.224594  1.305420 

Coefficients:
              Estimate Std. Error  t-value              Pr(>|t|)    
(Intercept)  -5.647995   0.727636  -7.7621   0.00000000000003498 ***
log(prbarr)  -0.677261   0.036368 -18.6226 < 0.00000000000000022 ***
log(prbconv) -0.513053   0.026102 -19.6555 < 0.00000000000000022 ***
log(prbpris)  0.123758   0.064823   1.9092             0.0567052 .  
log(avgsen)  -0.063690   0.053146  -1.1984             0.2312252    
log(polpc)    0.334638   0.030089  11.1217 < 0.00000000000000022 ***
log(wcon)    -0.031310   0.075292  -0.4158             0.6776698    
log(wtuc)    -0.068225   0.040288  -1.6934             0.0908801 .  
log(wtrd)     0.201556   0.084834   2.3759             0.0178118 *  
log(wfir)     0.033159   0.062842   0.5277             0.5979298    
log(wser)    -0.022411   0.043355  -0.5169             0.6053971    
log(wmfg)    -0.160275   0.072882  -2.1991             0.0282420 *  
log(wfed)     0.870285   0.141457   6.1523   0.00000000137643680 ***
log(wsta)    -0.420231   0.109188  -3.8487             0.0001312 ***
log(wloc)     0.138523   0.159416   0.8689             0.3852208    
---
Signif. codes:  0 '***' 0.001 '**' 0.01 '*' 0.05 '.' 0.1 ' ' 1

Total Sum of Squares:    206.38
Residual Sum of Squares: 78.749
R-Squared:      0.61843
Adj. R-Squared: 0.60974
F-statistic: 71.1972 on 14 and 615 DF, p-value: < 0.000000000000000222
\end{verbatim}

\begin{itemize}
\item
  \(\hat{\beta}_1\log(prbarr)\): Un aumento de 1\% en la probabilidad de
  arresto en disminuye en promedio 0.677\% la tasa de criminalidad,
  manteniendo todo lo demás constante.
\item
  \(\hat{\beta}_2\log(prbconv)\): Un aumento de 1\% en la probabilidad
  de condena disminuye en promedio un 0.513\% la tasa de criminalidad,
  manteniendo todo lo demás constante.
\item
  \(\hat{\beta}_3\log(prbconv)\): Un aumento de 1\% en la probabilidad
  de sentencia de prisión aumenta en promedio 0.124\% la tasa de
  criminalidad, manteniendo todo lo demás constante.
\end{itemize}

Los estimadores obtenidos a través del minimos cuadrados agrupados
podrían tener sesgo debido a varios factores:

\begin{itemize}
\item
  \textbf{\emph{Endogeneidad:}} Las variables de la probabilidad de
  arresto, condena y tiempo cumplido en prisión pueden tener correlación
  con el término de error. Los condados con altas tasas de criminalidad
  pueden tener más recursos u opciones para la aplicación de la ley, lo
  que aumentaría la probabilidad de ser arrestado o tener algún tipo de
  condena. Lo anterior podría generar una correlación positiva entre la
  probabilidad de arresto y el término de error, generando un sesgo en
  el estimador para la probabilidad de arresto.
\item
  \textbf{\emph{Efectos fijos no observados:}} El modelo tiene un
  término de caracteristicas o cualidades no observadas (\(a_i\)), que
  puede capturar elementos inobservables y constantes en el tiempo de
  cada lugar. Si estas cualidades tienen correlación con las variables
  de probabilidad usadas, los estimadores estarán sesgados. Por ejemplo,
  si los condados con leyes más estrictas hacia el crimen tienen una
  mayor probabilidad de ser arrestado, el parametro de \(log(prbarr)\) y
  el término \(a_i\) estarán correlacionados.
\item
  \textbf{\emph{Variables omitidas:}} Si existen variables relevantes
  que no se incluyen en el modelo, esto genera sesgo en los estimadores.
  Por ejemplo, si la tasa de desempleo afecta potencialmente a la tasa
  de criminalidad, pero no se incluye en el modelo, esto puede sesgar
  los estimadores de las variables incluidas en el modelo.
\end{itemize}

\subsection{\texorpdfstring{\textbf{\emph{Pregunta
2:}}}{Pregunta 2:}}\label{pregunta-2}

Estime ahora el modelo usando el método de primeras diferencias. ¿Existe
un cambio notable (respecto a lo hallado en 1) en el signo o en la
magnitud de los coeficientes de las variables asociadas a la justicia
penal? ¿Tiene un efecto disuasivo la justicia penal? ¿Son los efectos
estadísticamente significativos individualmente? Interprete los
resultados.

\begin{Shaded}
\begin{Highlighting}[]
\NormalTok{firstdiff }\OtherTok{\textless{}{-}} \FunctionTok{plm}\NormalTok{(}\FunctionTok{log}\NormalTok{(crmrte) }\SpecialCharTok{\textasciitilde{}} \FunctionTok{log}\NormalTok{(prbarr) }\SpecialCharTok{+} \FunctionTok{log}\NormalTok{(prbconv) }\SpecialCharTok{+} \FunctionTok{log}\NormalTok{(prbpris)  }\SpecialCharTok{+} 
                   \FunctionTok{log}\NormalTok{(avgsen) }\SpecialCharTok{+} \FunctionTok{log}\NormalTok{(polpc) }\SpecialCharTok{+} \FunctionTok{log}\NormalTok{(wcon) }\SpecialCharTok{+} \FunctionTok{log}\NormalTok{(wtuc) }\SpecialCharTok{+} 
                   \FunctionTok{log}\NormalTok{(wtrd) }\SpecialCharTok{+} \FunctionTok{log}\NormalTok{(wfir) }\SpecialCharTok{+} \FunctionTok{log}\NormalTok{(wser) }\SpecialCharTok{+} \FunctionTok{log}\NormalTok{(wmfg) }\SpecialCharTok{+} 
                   \FunctionTok{log}\NormalTok{(wfed) }\SpecialCharTok{+} \FunctionTok{log}\NormalTok{(wsta) }\SpecialCharTok{+} \FunctionTok{log}\NormalTok{(wloc), }\AttributeTok{data =}\NormalTok{ panel, }\AttributeTok{model =} \StringTok{"fd"}\NormalTok{)}
\FunctionTok{summary}\NormalTok{(firstdiff)}
\end{Highlighting}
\end{Shaded}

\begin{verbatim}
Oneway (individual) effect First-Difference Model

Call:
plm(formula = log(crmrte) ~ log(prbarr) + log(prbconv) + log(prbpris) + 
    log(avgsen) + log(polpc) + log(wcon) + log(wtuc) + log(wtrd) + 
    log(wfir) + log(wser) + log(wmfg) + log(wfed) + log(wsta) + 
    log(wloc), data = panel, model = "fd")

Balanced Panel: n = 90, T = 7, N = 630
Observations used in estimation: 540

Residuals:
      Min.    1st Qu.     Median    3rd Qu.       Max. 
-0.6660817 -0.0807718  0.0043541  0.0825722  0.6814652 

Coefficients:
               Estimate Std. Error  t-value              Pr(>|t|)    
(Intercept)   0.0013449  0.0137444   0.0978               0.92209    
log(prbarr)  -0.3450694  0.0306502 -11.2583 < 0.00000000000000022 ***
log(prbconv) -0.2524166  0.0185458 -13.6105 < 0.00000000000000022 ***
log(prbpris) -0.1769305  0.0265786  -6.6569      0.00000000007059 ***
log(avgsen)  -0.0080575  0.0220809  -0.3649               0.71533    
log(polpc)    0.3901473  0.0277500  14.0594 < 0.00000000000000022 ***
log(wcon)    -0.0411658  0.0314129  -1.3105               0.19061    
log(wtuc)     0.0111858  0.0134466   0.8319               0.40586    
log(wtrd)    -0.0411234  0.0317844  -1.2938               0.19630    
log(wfir)     0.0032492  0.0218791   0.1485               0.88200    
log(wser)     0.0165830  0.0147935   1.1210               0.26282    
log(wmfg)    -0.2297353  0.1020216  -2.2518               0.02475 *  
log(wfed)    -0.1797402  0.1708226  -1.0522               0.29319    
log(wsta)     0.1238491  0.0931275   1.3299               0.18413    
log(wloc)     0.0954568  0.1022299   0.9337               0.35086    
---
Signif. codes:  0 '***' 0.001 '**' 0.01 '*' 0.05 '.' 0.1 ' ' 1

Total Sum of Squares:    22.197
Residual Sum of Squares: 13.408
R-Squared:      0.39596
Adj. R-Squared: 0.37986
F-statistic: 24.5825 on 14 and 525 DF, p-value: < 0.000000000000000222
\end{verbatim}

\begin{itemize}
\item
  \(\hat{\beta}_1log(prbarr)\): Un aumento del 1\% en la probabilidad de
  arresto disminuye, en promedio, un 0.345\% la tasa de criminalidad,
  manteniendo constantes las demás variables en el modelo.
\item
  \(\hat{\beta}_2log(prbconv)\): Un aumento del 1\% en la probabilidad
  de condena disminuye, en promedio, un 0.252\% la tasa de criminalidad,
  manteniendo constantes las demás variables en el modelo.
\item
  \(\hat{\beta}_3log(prbpris)\): Un aumento del 1\% en la probabilidad
  de sentencia de prisión disminuye, en promedio, 0.177\% la tasa de
  criminalidad, manteniendo constantes las demás variables en el modelo.
\end{itemize}

Comparando los resultados entre el modelo de mínimos cuadrados agrupados
y el modelo de primeras diferencias, podemos observar algunos cambios en
los coeficientes de las variables asociadas a la justicia penal y en su
significancia estadística.

\begin{itemize}
\item
  \(log(prbarr)\): En el modelo de pooling, tiene un coeficiente
  negativo \emph{(-0.677)} y significativo a un nivel de confianza muy
  alto con un valor p cercano a cero, lo que sugiere que un aumento en
  la probabilidad de arresto se relaciona con una disminución en la tasa
  de criminalidad. Sin embargo, en el modelo de primeras diferencias, el
  signo es también negativo \emph{(-0.345)}, pero con una magnitud menor
  y sigue siendo significativo a un nivel muy alto con un valor p
  cercano a cero. Esto indica que el efecto disuasivo del arresto sobre
  la tasa de criminalidad se mantiene, pero quizás no es tan fuerte como
  en el modelo de pooling. La variable de ``probabilidad de arresto''
  (log(prbarr)) tiene un coeficiente negativo significativo en ambos
  modelos. Esto sugiere que un aumento en la probabilidad de arresto
  está asociado con una disminución en la tasa de criminalidad.
\item
  \(log(prbconv)\): Tanto en el modelo de pooling \emph{(-0.513)} como
  en el de primeras diferencias \emph{(-0.252)}, el estimador tiene un
  coeficiente negativo y significativo, lo que sugiere que un aumento en
  la probabilidad de condena se asocia con una disminución en la tasa de
  criminalidad para ambos modelos. Sin embargo, el coeficiente en el
  modelo de primeras diferencias es menor en magnitud que en el modelo
  de pooling. En ambos modelos, estos parametros indican que un aumento
  en la probabilidad de condena está asociado con una disminución en la
  tasa de criminalidad.
\item
  \(log(prbpris)\): En el modelo de pooling, la variable tiene un
  coeficiente positivo \emph{(0.124)} y significativo a un 1\%, lo que
  indica que un aumento en la probabilidad de sentencia de prisión
  podría estar relacionado con un aumento en la tasa de criminalidad. En
  el modelo de primeras diferencias, el coeficiente es negativo
  \emph{(-0.177)} y significativo con un valor p menor al nivel de
  significancia. Esto sugiere que un aumento en la probabilidad de
  sentencia de prisión está asociado con una disminución en la tasa de
  criminalidad, lo cual tiene mucho mas sentido.
\end{itemize}

En resumen, aunque los coeficientes de las variables asociadas a la
justicia penal tienen algunas diferencias en la magnitud y la
significancia estadistica, muestran un efecto disuasivo sobre la tasa de
criminalidad en ambos modelos. Estas diferencias podrían generarse
debido a la forma en que se controlan los efectos individuales en cada
modelo.

\subsection{\texorpdfstring{\textbf{\emph{Pregunta
3:}}}{Pregunta 3:}}\label{pregunta-3}

Vuelva a estimar el modelo pero ahora use efectos fijos en vez de
diferenciación. ¿Existe un cambio notable (respecto a lo hallado en 2)
en el signo o en la magnitud de los coeficientes de las variables
asociadas a la justicia penal? ¿Qué sucede con la significancia
estadística? Explique porqué existirían (o no) dichas diferencias.

\begin{Shaded}
\begin{Highlighting}[]
\NormalTok{fijos }\OtherTok{\textless{}{-}} \FunctionTok{plm}\NormalTok{(}\FunctionTok{log}\NormalTok{(crmrte) }\SpecialCharTok{\textasciitilde{}} \FunctionTok{log}\NormalTok{(prbarr) }\SpecialCharTok{+} \FunctionTok{log}\NormalTok{(prbconv) }\SpecialCharTok{+} \FunctionTok{log}\NormalTok{(prbpris)  }\SpecialCharTok{+} 
               \FunctionTok{log}\NormalTok{(avgsen) }\SpecialCharTok{+} \FunctionTok{log}\NormalTok{(polpc) }\SpecialCharTok{+} \FunctionTok{log}\NormalTok{(wcon) }\SpecialCharTok{+} \FunctionTok{log}\NormalTok{(wtuc) }\SpecialCharTok{+} \FunctionTok{log}\NormalTok{(wtrd) }\SpecialCharTok{+} 
               \FunctionTok{log}\NormalTok{(wfir) }\SpecialCharTok{+} \FunctionTok{log}\NormalTok{(wser) }\SpecialCharTok{+} \FunctionTok{log}\NormalTok{(wmfg) }\SpecialCharTok{+} \FunctionTok{log}\NormalTok{(wfed) }\SpecialCharTok{+} \FunctionTok{log}\NormalTok{(wsta) }\SpecialCharTok{+} 
               \FunctionTok{log}\NormalTok{(wloc), }\AttributeTok{data =}\NormalTok{ panel, }\AttributeTok{model =} \StringTok{"within"}\NormalTok{)}
\FunctionTok{summary}\NormalTok{(fijos)}
\end{Highlighting}
\end{Shaded}

\begin{verbatim}
Oneway (individual) effect Within Model

Call:
plm(formula = log(crmrte) ~ log(prbarr) + log(prbconv) + log(prbpris) + 
    log(avgsen) + log(polpc) + log(wcon) + log(wtuc) + log(wtrd) + 
    log(wfir) + log(wser) + log(wmfg) + log(wfed) + log(wsta) + 
    log(wloc), data = panel, model = "within")

Balanced Panel: n = 90, T = 7, N = 630

Residuals:
       Min.     1st Qu.      Median     3rd Qu.        Max. 
-0.58697823 -0.07292538 -0.00046989  0.07755788  0.51724500 

Coefficients:
               Estimate Std. Error  t-value              Pr(>|t|)    
log(prbarr)  -0.3859180  0.0324015 -11.9105 < 0.00000000000000022 ***
log(prbconv) -0.3036104  0.0211703 -14.3413 < 0.00000000000000022 ***
log(prbpris) -0.1919037  0.0325306  -5.8992        0.000000006544 ***
log(avgsen)   0.0240314  0.0252588   0.9514             0.3418364    
log(polpc)    0.4300293  0.0268068  16.0418 < 0.00000000000000022 ***
log(wcon)    -0.0330217  0.0391582  -0.8433             0.3994497    
log(wtuc)     0.0287136  0.0177099   1.6213             0.1055464    
log(wtrd)    -0.0393415  0.0412091  -0.9547             0.3401785    
log(wfir)    -0.0129121  0.0286998  -0.4499             0.6529656    
log(wser)     0.0040193  0.0194285   0.2069             0.8361861    
log(wmfg)    -0.3577230  0.0995598  -3.5930             0.0003575 ***
log(wfed)    -0.5765660  0.1620628  -3.5577             0.0004080 ***
log(wsta)     0.1802300  0.0954239   1.8887             0.0594780 .  
log(wloc)     0.3282463  0.1019427   3.2199             0.0013615 ** 
---
Signif. codes:  0 '***' 0.001 '**' 0.01 '*' 0.05 '.' 0.1 ' ' 1

Total Sum of Squares:    17.991
Residual Sum of Squares: 10.411
R-Squared:      0.42132
Adj. R-Squared: 0.308
F-statistic: 27.3541 on 14 and 526 DF, p-value: < 0.000000000000000222
\end{verbatim}

\begin{itemize}
\item
  \(\hat{\beta}_1\stackrel{..}{\log(prbarr)}\): Un aumento del 1\% en la
  probabilidad de arresto disminuye, en promedio, un 0.386\% la tasa de
  criminalidad, manteniendo constantes las demás variables en el modelo.
\item
  \(\hat{\beta}_2\stackrel{..}{\log(prbconv)}\): Un aumento del 1\% en
  la probabilidad de condena disminuye, en promedio, un 0.304\% la tasa
  de criminalidad, manteniendo constantes las demás variables en el
  modelo.
\item
  \(\hat{\beta}_3\stackrel{..}{\log(prbpris)}\): Un aumento del 1\% en
  la probabilidad de sentencia de prisión disminuye, en promedio,
  0.192\% la tasa de criminalidad, manteniendo constantes las demás
  variables en el modelo.
\end{itemize}

Al realizar la comparación de los resultados entre el modelo de primeras
diferencias y el modelo de efectos fijos, se observan algunos cambios en
los coeficientes de las variables de probabilidad asociadas a la
justicia penal y mantienen su significancia estadística.

\begin{itemize}
\item
  \(log(prbarr)\): En el modelo de primeras diferencias, tiene un
  coeficiente negativo \emph{(-0.345)} y significativo con un p valor
  muy cercano a cero, indicando un efecto disuasivo de la probabilidad
  de arresto sobre la tasa de criminalidad. Por otra parte, en el modelo
  de efectos fijos el coeficiente es también negativo y significativo
  \emph{(-0.386)}, pero la magnitud es ligeramente menor que en el
  modelo de primeras diferencias.
\item
  \(log(prbconv)\): En ambos modelos, tiene un coeficiente negativo y
  significativo. Para modelo de diferencias \emph{(-0.252)}, indicando
  un efecto disuasivo de la condena sobre la tasa de criminalidad. Sin
  embargo, la magnitud del coeficiente es menor en el modelo de efectos
  fijos \emph{(-0.304)}. Aunque la magnitud es menor que en el modelo de
  primeras diferencias los estimadores de ambos modelos sugieren que, un
  aumento en la probabilidad de tener condena luego de ser arrestado
  disminuye la tasa de criminalidad.
\item
  \(log(prbpris)\): En el modelo de primeras diferencias el coeficiente
  es negativo y significativo \emph{(-0.177)}. Por otra parte, en el
  modelo de efectos fijos el coeficiente es negativo y significativo
  \emph{(-0.192)}. Aunque la magnitud es menor que en el modelo de
  primeras diferencias los estimadores de ambos modelos sugieren que, un
  aumento en la probabilidad de ir a prisión disminuye la tasa de
  criminalidad.
\end{itemize}

Por otra parte, en ambos modelos, la mayoría de los coeficientes de las
variables asociadas a la justicia penal son estadísticamente
significativos a un nivel de confianza muy alto, dado que el valor p es
menor al nivel de significancia (p-valor \textless{} 0.001).

\subsection{\texorpdfstring{\textbf{\emph{Pregunta
4:}}}{Pregunta 4:}}\label{pregunta-4}

Realice un test de significancia estadística de los efectos fijos no
observables \(a_i\). Para ello use el comando
pFtest(modelo\_efecto\_fijo, modelo\_pooling), el cual realiza el test F
discutido en clase (requiere como input los modelos estimados en 1 y en
3).

\[Hipotesis\ nula: H_0 : a_i = 0 \]
\[Hipotesis\ alternativa: H_1 : Al\ menos\ uno\ es\ significativo\]

Si el valor p es menor que el nivel de significancia, se rechaza la
hipótesis nula.

\begin{Shaded}
\begin{Highlighting}[]
\FunctionTok{pFtest}\NormalTok{(fijos, pooled)}
\end{Highlighting}
\end{Shaded}

\begin{verbatim}

    F test for individual effects

data:  log(crmrte) ~ log(prbarr) + log(prbconv) + log(prbpris) + log(avgsen) +  ...
F = 38.794, df1 = 89, df2 = 526, p-value < 0.00000000000000022
alternative hypothesis: significant effects
\end{verbatim}

El resultado arroja un valor-p muy pequeño de \emph{2,2e-16}, lo que
indica que existe una fuerte evidencia estadistica para rechazar la
hipótesis nula. Finalmente se concluye que al menos uno de los efectos
individuales es significativo con un \(\alpha\) de \emph{0.05}, es decir
que el modelo en el cual se utiliza el metodo de los efectos fijos
mejora significativamente el modelo en comparación con el modelo de
mínimos cuadrados agrupados.

\subsection{\texorpdfstring{\textbf{\emph{Pregunta
5:}}}{Pregunta 5:}}\label{pregunta-5}

Vuelva a estimar el modelo pero ahora use efectos aleatorios. ¿Existe un
cambio notable (respecto a lo hallado en 3) en el signo o en la magnitud
de los coeficientes de las variables asociadas a la justicia penal? Que
conclusión podría sacar de la comparación de los modelos de efectos
fijos y efectos aleatorios?

\begin{Shaded}
\begin{Highlighting}[]
\NormalTok{aleatorios }\OtherTok{\textless{}{-}} \FunctionTok{plm}\NormalTok{(}\FunctionTok{log}\NormalTok{(crmrte) }\SpecialCharTok{\textasciitilde{}} \FunctionTok{log}\NormalTok{(prbarr) }\SpecialCharTok{+} \FunctionTok{log}\NormalTok{(prbconv) }\SpecialCharTok{+} \FunctionTok{log}\NormalTok{(prbpris)  }\SpecialCharTok{+} 
                    \FunctionTok{log}\NormalTok{(avgsen) }\SpecialCharTok{+} \FunctionTok{log}\NormalTok{(polpc) }\SpecialCharTok{+} \FunctionTok{log}\NormalTok{(wcon) }\SpecialCharTok{+} \FunctionTok{log}\NormalTok{(wtuc) }\SpecialCharTok{+} 
                    \FunctionTok{log}\NormalTok{(wtrd) }\SpecialCharTok{+} \FunctionTok{log}\NormalTok{(wfir) }\SpecialCharTok{+} \FunctionTok{log}\NormalTok{(wser) }\SpecialCharTok{+} \FunctionTok{log}\NormalTok{(wmfg) }\SpecialCharTok{+} \FunctionTok{log}\NormalTok{(wfed) }\SpecialCharTok{+} 
                    \FunctionTok{log}\NormalTok{(wsta) }\SpecialCharTok{+} \FunctionTok{log}\NormalTok{(wloc), }\AttributeTok{data =}\NormalTok{ panel, }\AttributeTok{model =} \StringTok{"random"}\NormalTok{)}
\FunctionTok{summary}\NormalTok{(aleatorios)}
\end{Highlighting}
\end{Shaded}

\begin{verbatim}
Oneway (individual) effect Random Effect Model 
   (Swamy-Arora's transformation)

Call:
plm(formula = log(crmrte) ~ log(prbarr) + log(prbconv) + log(prbpris) + 
    log(avgsen) + log(polpc) + log(wcon) + log(wtuc) + log(wtrd) + 
    log(wfir) + log(wser) + log(wmfg) + log(wfed) + log(wsta) + 
    log(wloc), data = panel, model = "random")

Balanced Panel: n = 90, T = 7, N = 630

Effects:
                  var std.dev share
idiosyncratic 0.01979 0.14069 0.193
individual    0.08277 0.28770 0.807
theta: 0.8182

Residuals:
     Min.   1st Qu.    Median   3rd Qu.      Max. 
-0.898669 -0.078049  0.016534  0.094230  0.532297 

Coefficients:
               Estimate Std. Error  z-value              Pr(>|z|)    
(Intercept)  -0.8675192  0.5899323  -1.4705              0.141415    
log(prbarr)  -0.4613921  0.0326700 -14.1228 < 0.00000000000000022 ***
log(prbconv) -0.3559045  0.0214099 -16.6233 < 0.00000000000000022 ***
log(prbpris) -0.1954782  0.0350384  -5.5790          0.0000000242 ***
log(avgsen)   0.0246280  0.0274570   0.8970              0.369737    
log(polpc)    0.4322964  0.0272689  15.8531 < 0.00000000000000022 ***
log(wcon)    -0.0243095  0.0422322  -0.5756              0.564875    
log(wtuc)     0.0227704  0.0193170   1.1788              0.238487    
log(wtrd)    -0.0152818  0.0447416  -0.3416              0.732684    
log(wfir)    -0.0105343  0.0312808  -0.3368              0.736293    
log(wser)    -0.0037376  0.0211780  -0.1765              0.859913    
log(wmfg)    -0.2574320  0.0915891  -2.8107              0.004943 ** 
log(wfed)    -0.0239451  0.1556607  -0.1538              0.877745    
log(wsta)    -0.0643540  0.0950754  -0.6769              0.498486    
log(wloc)     0.2014455  0.1066653   1.8886              0.058949 .  
---
Signif. codes:  0 '***' 0.001 '**' 0.01 '*' 0.05 '.' 0.1 ' ' 1

Total Sum of Squares:    24.214
Residual Sum of Squares: 14.649
R-Squared:      0.39501
Adj. R-Squared: 0.38124
Chisq: 401.544 on 14 DF, p-value: < 0.000000000000000222
\end{verbatim}

\begin{itemize}
\item
  \(\hat{\beta}_1\overset{\sim}{\log(prbarr)}\): Un aumento del 1\% en
  la probabilidad de arresto disminuye, en promedio, un 0.461\% la tasa
  de criminalidad, manteniendo constantes las demás variables en el
  modelo.
\item
  \(\hat{\beta}_2\overset{\sim}{\log(prbconv)}\): Un aumento del 1\% en
  la probabilidad de condena disminuye, en promedio, un 0.355\% la tasa
  de criminalidad, manteniendo constantes las demás variables en el
  modelo.
\item
  \(\hat{\beta}_3\overset{\sim}{\log(prbpris)}\): Un aumento del 1\% en
  la probabilidad de sentencia de prisión disminuye, en promedio,
  0.195\% la tasa de criminalidad, manteniendo constantes las demás
  variables en el modelo.
\end{itemize}

Al comparar los resultados del modelo de efectos aleatorios con los
modelos de efectos fijos y de primeras diferencias, podemos observar
algunas diferencias en los coeficientes de las variables asociadas a la
justicia penal.

\(log(prbarr)\): En el modelo de efectos aleatorios el coeficiente
estimado es \emph{(-0.461)}. Por otra parte, en el modelo de efectos
fijos el coeficiente estimado es \emph{(-0.385)}. Ambos modelos son
significativos.

\(log(prbconv)\): En el modelo de efectos aleatorios el coeficiente
estimado es \emph{(-0.355)}. Por otra parte, el modelo de efectos fijos
el coeficiente estimado es \emph{(-0.303)}.

\(log(prbpris)\): En el modelo de efectos aleatorios el coeficiente
estimado es \emph{(-0.195)}. Por otra parte, en el modelo de efectos
fijos el coeficiente estimado es \emph{(-0.191)}.

Como sabemos ambos modelos son insesgados y consistentes, pero no cual
es mas eficiente sin hacer un test de Hausman En el modelo de efectos
aleatorios, los coeficientes tienen magnitudes y significancias
estadísticas similares a los del modelo de efectos fijos, lo que indica
que los efectos aleatorios no alteran sustancialmente las conclusiones
sobre la relación entre las variables asociadas a la justicia penal y la
tasa de criminalidad. Además, en ambos modelos, un aumento en la
probabilidad de arresto, condena y sentencia de prisión se asocia con
una disminución en la tasa de criminalidad.

\subsection{\texorpdfstring{\textbf{\emph{Pregunta
6:}}}{Pregunta 6:}}\label{pregunta-6}

Realice el test de Hausman. Para ello use el comando
phtest(modelo\_efecto\_fijo, modelo\_efecto\_aleatorio), el cual realiza
el test de Hausman discutido en clase (requiere como input los modelos
estimados en 3 y en 5). ¿Cuál es el mejor estimador, el de efectos fijos
o el de efectos aleatorios?

El test de Hausman se utiliza para determinar si es más apropiado
utilizar un modelo de efectos fijos o un modelo de efectos aleatorios.

\[Hipotesis\ nula: H_0 : Cov(x_{kit},a_i ) = 0 ,\  \forall k\]
\[Hipotesis\ alternativa: H_1 : Cov(x_{kit},a_i ) \neq 0, \ para \ algún \ k \]

\begin{Shaded}
\begin{Highlighting}[]
\FunctionTok{phtest}\NormalTok{(fijos, aleatorios)}
\end{Highlighting}
\end{Shaded}

\begin{verbatim}

    Hausman Test

data:  log(crmrte) ~ log(prbarr) + log(prbconv) + log(prbpris) + log(avgsen) +  ...
chisq = 143.58, df = 14, p-value < 0.00000000000000022
alternative hypothesis: one model is inconsistent
\end{verbatim}

El resultado arroja un valor-p muy pequeño de \emph{2,2e-16}, lo que
indica que existe una fuerte evidencia estadistica para rechazar la
hipótesis nula. Finalmente, se concluye que uno de los 2 modelos es
inconsistente (Efecto Aleatorio), esto quiere decir que el alpha
correlaciona con las variables, por lo que es mejor utilizar el modelo
de efectos fijos, debido a a que su estimador va a ser insesgado y
consistente en muestras grandes.

\end{document}
